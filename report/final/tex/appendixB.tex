\section{Developer Documentation}\label{sec:devdoc}


\subsection{Build environment}

The project’s codebase can be configured with CMake \footnote{https://cmake.org/} for development on Linux and Windows. CMake is a platform independent build and freely available tool for C++, that automatically generates compiler configurations. On linux systems, the Makefiles, that are used by the GNU compiler chain, and on Microsoft Windows, Visual Studio project files can be generated from the given configurations. This eliminates the need for tracking such files for different platforms or different versions of development environments.
\\
\\
CMake comes with either command line or GUI tools to configure the concrete setup. The main configuration is defined in the file CMakeLists.txt in the root code folder of the application. Using one of the GUI-tools for configuration for example, the CMakeLists.txt has to be loaded. The configuration then provides variables to set the path to the dependencies. When the configuration is finished, the Makefiles or Visual Studio project files can be generated.


\subsection{Dependencies}

The application has dependencies on OpenCV and the boost c++ libraries. Both libraries have to be installed and added to the system's library path variable. There are non-commercial licenses for both dependencies, so they are freely available and usable without infringement.
\\
\\
OpenCV \footnote{http://opencv.org/} is used in the version 2.4. OpenCV version 3 is not supported, as there are major API changes making it incompatible. The reason for not choosing the latest version was, that this version was released too shortly before the start of this project and has not been in a stable and coherent state at that time regarding the API and the official documentation. OpenCV provides the image processing and machine learning algorithms and techniques used in the application. As it is also configured with CMake, the dependency is configured by simply providing the CMake configuration files of OpenCV, usually located in the root OpenCV installation folder.
\\
\\
The boost c++ libraries \footnote{http://www.boost.org/} are used in the version 1.59.0. As these libraries are downward compatible, later versions can also be used. The boost c++ libraries provide the mechanisms used for file system handling like reading and writing data to and from the file system and for the configuration handling. Though most parts of the libraries are header-only, the file system module requires to either build the libraries or install a complete package containing built libraries. In the CMake configuration, the path to the include files folder and to the installed libraries folder has to be provided to enable the build configuration to set up the Makefiles or Visual Studio project files correctly.


\subsection{Automated HMTL API Documentation Generation}

The application's code contains annotations and comments that can be processed by Doxygen \footnote{http://www.doxygen.org} to automatically generate API documentation in HTML. A corresponding configuration file is located in the root code folder, called "`Doxyfile"', providing all necessary information to the generator. The generator is invoked from the root code folder as working directory, as the local "`src"' folder is the base for the source files, that have to be processed. Giving the configuration file "`Doxyfile"' as parameter, the generator will put the generated documentation into a local folder called "`doc"'. The "`index.html"' file in the "`doc"' folder is the starting reference for the documentation. An example call from the command line executed in the root code folder of the application looks like this:
\begin{verbatim}
./doxygen Doxyfile
\end{verbatim}


\subsection{Implementation Details}

TODO
\\
Classes
\\
Program flow
%\begin{figure}
\begin{tikzpicture}[-latex][H]
  \matrix (chart)
    [
      matrix of nodes,
      column sep      = 3em,
      row sep         = 5ex,
      column 1/.style = {nodes={decision}},
      column 2/.style = {nodes={env}}
    ]
    {
      |[treenode]| Load images           &                \\
     |[treenode]| Add flipped images                &        \\
      Detect faces?               & Perform face detection       \\
      Apply PCA?                 & Apply PCA feature reduction         \\
     |[treenode]| Compute HOG  & \\
     |[treenode]| Train SVM &        \\
      |[treenode]| Predict validation data  &        \\
    %   & & |[decision]| numbered? \\
    %   & & |[treenode]| Add a \texttt{*} & |[finish]| Done \\
    };
    
  \draw
  (chart-1-1) edge (chart-2-1)
    (chart-2-1) edge (chart-3-1)
    \foreach \x/\y in {3/4, 4/5} {
      (chart-\x-1) \no (chart-\y-1) 
     }
    \foreach \x in {3,...,4} {
      (chart-\x-1) \yes (chart-\x-2)
    }
    (chart-3-2) edge (chart-4-1)
    (chart-4-2) edge (chart-5-1)
    
    (chart-5-1) edge (chart-6-1)
    (chart-6-1) edge (chart-7-1)
    ;

\end{tikzpicture}
\caption{Application flow chart}
\label{figure:flowchart}

\end{figure}
\\
\\
The data module provides data types and data structures for the application. The class DataSet is the basic data type. It holds the images each together with the file name and the feature vectors extracted from that image. In the course of the main processing pipeline, the images themselves are deleted after the feature extraction step to free memory. The class furthermore provides a method to split the images and respectively their feature vectors into two parts of definable size for training and validation purpose. The class TrainingData holds the training and label matrix needed by the SVM for training in OpenCV matrix format. The constructor creates these matrices from the given vectors of positive and negative training samples, i.e. feature vectors.
\\
\\
The module IOUtils provides static methods for file system handling, input and output of images and basic image operations. There are methods to load a complete catalog of images and methods to load single images or all images contained in a folder. In a catalog of images, the images have to be organized in sub folders according to their categories and the sub folder names define the category names. There are methods to load and save SVMs from and to the file system. Furthermore, basic image operations like histogram equalization and adding flipped versions of images are part of this module as this functionality is integrated into the loading process to avoid memory issues due to allocation of images multiple times during the according process.
\\
TODO this structure needs to be changed in code, move stuff to utils!
\\
\\
The featureExtraction module provides the functionality for the HOG feature extraction. The class umHOG provides static methods to compute the HOG features for a DataSet object or a single image in OpenCV matrix format. It makes use of the OpenCV HOGDescriptor class to accomplish this task. The feature vectors are of the type vector of floats. The parameters for the extraction process correspond to those needed for the OpenCV HOGDescriptor extraction method. Furthermore the class provides a method to visualize the extracted gradients together with the original image. This method  has been enhanced to support generic input sizes and print out concrete values for gradient average or distinct gradients per extracted cell. The class umPCA provides static methods to reduce the dimensionality of the feature space. It makes use of the OpenCV PCA class and applies the Principal Component Analysis on the cells of the HOG features reducing the dimensionality to the number of gradient bins per cell.
%\footnote{http://www.juergenwiki.de/work/wiki/doku.php?id\=public:hog\_descriptor\_computation\_and\_visualization\#computing\_the\_hog\_descriptor\_using\_opencv}
\\
\\
The svm module provides the functionality for the machine learning parts of the processing pipeline. The class umSVM holds an object of OpenCV CvSVM and provides methods to train this SVM instance, to predict data and load and save this instance. The SVM instance is of type linear SVM C\_SVC, which can be configured using the SVM C value to refine the hyperplane on imperfect separation. The class ClassificationSVM holds a map of categorized umSVM instances and provides necessary methods to interact with them. These methods include training the SVMs on the complete catalog of images, loading and saving categorized SVMs, and using the SVM instances for prediction.
\\
\\
TODO needs to be moved in code !!
\\
The class CatalogeTraining defines the main processing pipeline for training categorized SVMs with a catalog of categorized images. The processing pipeline for single patch images, i.e. manual extracted patches, is setup like this:

\begin{enumerate}
	\item Loading an image catalog from disc.
	\item Refine the images through histogram equalization.
	\item Double the sample size by adding flipped versions (Optional).
	\item Compute the HOG features for the images.
	\item Setup training and validation data.
	\item Train the SVMs on the training data.
	\item Validate the SVMs through prediction of the validation data.
	\item Exchange training and validation data and retrain the SVMs (Optional).
\end{enumerate}

In the course of the process, according methods of the application modules are invoked. The class itself handles the data setup. The class holds a map of categorized DataSets and maps for categorized TrainingData for training and validation purpose. It provides methods to collect positive and negative training samples, from which the TrainingData instances are constructed. These methods ensure equality of the size of positive and negative samples and an even distribution of negative samples among all other categories apart from the current positive one as far as the category sizes permit such equality and even distribution.
\\
\\
The class Configuration holds the configurable parameters for the application. The parameters are loaded from a given configuration file on instance construction. The class Constants holds the application parameters, that are not supposed to change during normal usage. This includes standard values for the SVM labels and debug flags for debugging the application.