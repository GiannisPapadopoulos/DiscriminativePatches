
\section{Developed Application for Image Classification}\label{sec:application}

This section presents the application that has been developed. The application can be used to train an SVM for each category of a catalog of images or predict images using already trained SVMs. For training, it loads a set of images and trains SVMs for the given categories. The classifiers work on the HOG features extracted from the input images. The steps of the algorithm that performs the image classification are described in a general manner. Details of the implementation can be found in the appendix section \ref{sec:devdoc}.
\\
\\
To train classifiers, the application loads a catalog of images from the file system. At least 2 different categories have to be provided in separate sub folders. The folder names denote the category labels. If it is desired to obtain more data samples to enhance the training of the classifiers, additional images from the horizontally flipped versions of the input images are computed and added to the input data. This doubles the original sample size and eases the problem of overfitting. To improve the quality of the images in regard of the performed feature extraction, histogram equalization is applied to the images as a preprocessing step of the feature extraction. This increases the contrast of the images and produces sharper edges for the edge gradient computation of the feature extractor. The feature extraction is performed using Histogram of Oriented Gradients. The computed features are split into defined parts for training and validation purpose. If there are multiple samples from a similar source contained in the dataset, the separation will make sure that they are not distributed across the training and validation data. For each category, a training and a validation matrix is built each by the corresponding elements of the category itself as positive samples and an equal number of evenly distributed samples from all other corresponding categories as negative samples. The training matrix is used to train a classifier for each category. The validation matrix is used to test and validate the effectiveness of the trained classifier with a large number of unknown samples. Cross validation can be performed by exchanging the training and the validation matrix to enhance the SVMs with additional data. Finally the trained classifiers can be saved to disc for further usage. There will be one classifier for each category.
