\section{Conclusion and Future Work}

The paper described the work and results of the research project on Mid-level Discriminative Patches. An application has been developed making use of the main parts of the algorithm proposed by Singh et Al. \cite{Singh2012DiscPat} namely the feature extraction using HOG and the classifier using linear SVM. Preprocessing of the data applying histogram equalization ensured good results on the feature extraction. Special care was taken with regard to the setup of the training data in terms of equality in the size of the positive and negative samples and the even distribution of the negative samples among all available negative categories. The acquired data was carefully chosen to contain expressive images to accomplish the task of classifying emotional expressions. Proper separation of the image series in combination with the possibility to add additional images on application runtime ensured the maximum size of training samples available.
\\
\\
From the results of the experiments, it is clear that among all the manually extracted patches, mouth patch is the most representative and distinguishable. The face, as a whole patch, is high-level patch and has too much information. Results of the experiments performed on whole face patches are poorer than mid-level (manually extracted) patches. 

  