
\begin{abstract}

TODO: same as introduction, change one of them !!
\\
\\
Unsupervised Discovery of mid-level discriminative patches is a method to extract the primitives for visual information from an image.  Singh et al. elaborated on this method by stating that the primitives have to satisfy the requirements of being representative as well as discriminative and having been discovered in a fully unsupervised manner \cite{Singh2012DiscPat}. The method combines techniques from the fields of computer vision and machine learning to produce promising results in image recognition and classification. 
\\
\\
The aim of this project is to apply this method on the classification of images of human facial expressions. This can, for example, aid in human-computer interaction by recognizing particular emotions or cognitive states through visual or expressive features contained in an image. Then computers are enabled to react and interact with the counterpart in an appropriate manner. As a result of the project's work an application has been developed for classification of human emotions and the effectiveness of the image recognition by this classification will be proven with selected, distinct features of various images showing different facial expressions.
\end{abstract}
