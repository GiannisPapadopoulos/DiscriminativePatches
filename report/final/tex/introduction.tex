
\section{Introduction}\label{sec:Introduction}

TODO: Give an intorduction to the problem and an outline of the work. Something that I miss (at least in this version of the report) is the rationale behind the original work: that the idea is to let the system detect itself useful patches. Then, it would be nice to also have a connection with k-means and SVM, in terms of that
\\
\\
Unsupervised Discovery of mid-level discriminative patches is a method to extract the primitives for visual information from an image.  Singh et al. elaborated on this method by stating that the primitives have to satisfy the requirements of being representative as well as discriminative and having been discovered in a fully unsupervised manner \cite{Singh2012DiscPat}. The method combines techniques from the fields of computer vision and machine learning to produce promising results in image recognition and classification. 
\\
\\
The aim of this project is to apply this method on the classification of images of human facial expressions. This could, for example, aid in human-computer interaction by recognizing particular emotions or cognitive states through visual or expressive features contained in an image. The purpose of this project is to determine classifiers for human facial expressions by delivering an extensible, easy-to use and well documented code base using C++ as programming language and the image processing library OpenCV as toolset for the basic techniques and algorithms needed to implement the functionality. This paper describes the results of the research project by first giving an overview of the technical background and describing the method implemented by Singh et Al. \cite{Singh2012DiscPat}. Then the project work is presented and finally an outlook on possible future work is given.