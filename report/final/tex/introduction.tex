
\section{Introduction}\label{sec:Introduction}
Unsupervised Discovery of mid-level discriminative patches is a method to extract from an image the primitives for visual information that satisfy the requirements of being representative as well as discriminative and that have been discovered in a fully unsupervised manner. The method was elaborated by Singh et Al. at the Carnegie Mellon University of Pittsburgh, Pennsylvania, and published at the European Conference on Computer Vision in the year 2012.\cite{Singh2012DiscPat} The method is settled in the fields of computer vision and machine learning and combines techniques from these fields to produce promising results in image recognition and classification. Singh et Al. showed the capabilities of their method in a follow-up publication What Makes Paris Look like Paris?.\cite{doersch2012what} Here they identified the most distinct elements on images of architecturally different cities like Paris and London classifying these images as being shot in the corresponding city.
\\
\\
The idea of this project is to apply this method on the classification of images of human facial expressions. This could for example aid in human computer interaction to recognize particular emotions or cognitive states through visual or expressive features contained in an image.
\\
Put in one phrase the goal of this project is to find classifiers for human facial expressions.
In detail the requirements are to deliver an extensible, easy-to use and well documented code base using C++ as programming language and the image processing library OpenCV as toolset for the basic techniques and algorithms needed to implement the functionality.
\\
\\
This paper describes the results of the research project by first giving an overview of the technical background and describing the method implemented by Singh et Al.. Then the project work is presented and finally and outlook on possible future work is given.