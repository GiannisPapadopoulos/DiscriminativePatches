\appendix
\appendixpage
\addappheadtotoc

\section{User Documentation}\label{sec:userdoc}

The application has three operating modes: train, retrain and predict. In train-mode, new SVMs are trained from given input data. In retrain-mode, already trained SVMs are loaded to enhance them with additional data. In predict-mode, trained SVMs are loaded and used to classify given input data.


\subsection{How to start the application}
The application is implemented as a commandline application. It is invoked on the commandline by it's name "`MAIProject"' with the filename and path to the configuration file as parameter. The parameter option is denoted as "`-config"'. The following example shows how to call the executable from the current directory assuming the configuration file config.ini is located in the same place.
\begin{verbatim}
./MAIProject -config config.ini
\end{verbatim}

%\lstset{language=bash}
%\begin{lstlisting}
%./MAIProject -config config.ini
%\end{lstlisting}

%\texttt{./MAIProject -config config.ini} 


\subsection{How to configure the application}
The application is configured in detail by a configuration file. This configuration file is formatted using the ini-format. This format follows a simple syntax of key/value pairs noted in one line separated by an equal sign. Sections allow a simple structuring of the key/value pairs. Their names are denoted in square brackets. Section names have to be unique per file and keys have to be unique per section. The file itself is plain text. For this application, there are sections for each part of the implemented algorithm giving detailed control of the configurable aspects of the corresponding feature.
\\
\\
The section MAIN defines the application's operating mode with the parameter MODE. This parameter can have one of the following values: TRAIN, RETRAIN or PREDICT. If the mode set to RETRAIN, the parameter SVM\_FILEPATH defines the location of the already trained SVMs that should be loaded for further training. The data needed for this is configured in the DATA section. In predict-mode, the parameter SVM\_FILEPATH defines the location of the trained SVMs used to classify images. These images are loaded from the location defined in parameter IMAGE\_FILEPATH. This parameter can either define a single image file or a folder containing images. In the latter case, all of the images contained in that folder will be classified.
\\
\\
The section DATA provides parameters to configure the input dataset. The parameter FILEPATH defines the location of the image catalog that should be used for training. The parameter DATASET\_DIVIDER defines the size of the validation part of the input dataset by dividing the size of the whole dataset, e.g. a value of 2 means that 1/2 of the input will be used for validation and 1/2 for training of the SVMs. The flag ADD\_FLIPPED\_IMAGES enables the addition of horizontally flipped versions of the input images to effectively double the overall size of samples, if this is necessary to obtain more training samples.
\\
\\
The section HOG provides parameters to configure the size of image, block and cells, the block stride and the number of bins used to extract the HOG features from the input images. If it is desired to visualize the HOG features together with the image they are computed from, the flag WRITE\_HOGIMAGES can be set together with parameters defining the scaling of the image and the visualized gradients. Concrete values depend on the original image size and the parameters used to compute the features.
\\
\\
The section SVM provides parameters used in combination with the vector machines. The C\_VALUE parameter is used to penalize outliers on an imperfect separation. The flag PREDICT\_TRAININGDATA enables or disables prediction of the dataset used for training. The flag WRITE\_SVM can be set if the trained SVMs should be saved to disc in combination with the parameter FILEPATH, which defines the output location of the saved SVMs. The flag CROSS\_VALIDATE can be set to swap training and validation data and further train the SVMs with the validation data.
\\
\\
The section FACE\_DETECTION provides parameters to configure the haar cascade classifier. The flag DETECT\_FACES enables face detection on the images of the input dataset. The parameter FILENAME defines the path and filename of the trained haar cascade classifier performing the face detection itself. Such classifiers can be found in the corresponding OpenCV package. The parameters MIN\_SIZE and MAX\_SIZE indicate the minimum and maximum possible size of the obtained faces. They depend on the image size of the input dataset which has to be examined to obtain rational values for these parameters.
